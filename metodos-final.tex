\documentclass[a4paper,spanish]{article}

\usepackage[spanish,activeacute]{babel}
\usepackage{moreverb}
\usepackage{fancyhdr}
\usepackage{graphicx}
\usepackage{multicol}
\usepackage{theorem,amsmath,amssymb,latexsym}
\usepackage{enumerate}
\usepackage{color}

\oddsidemargin -0.5in
\textwidth 7.2in
\topmargin 0in
\addtolength{\topmargin}{-.5in}
\textheight 10in
\parskip=2ex
\pagestyle{fancy}
\parindent=0cm
%usar el segundo nivel de enumeracion con letras
%\Rnewcommand{\labelenumii}{\alph{enumii}. }

\newcommand{\Rv}{\marginpar{\textcolor{red}{$|$}}}
\newcommand{\nohecho}{\marginpar{NO HECHO}}
\newcommand{\note}[1]{\textcolor{red}{\textbf{#1}} \Rv}

%espaciado
\newcommand{\vsp}{\vspace{0.4cm}}
\newcommand{\hsp}{\hspace*{0.12cm}}

%comandos para el resumen
\newcommand{\tab}{\hspace*{1cm}}
\newcommand{\llamada}[1]{\begin{center} \bfseries #1 \mdseries \end{center}}
\newcommand{\nota}[1]{\vsp\defi{Nota}{#1}\vsp}
\newcommand{\R}[0]{\mathbb{R}}
\newcommand{\N}[0]{\mathbb{N}}
\newcommand{\norma}[1]{\left\|#1\right\|}
\newcommand{\limite}[2]{\lim_{ #1 \rightarrow #2}}
\newcommand{\D}[0]{\mathbf{D}}
\newcommand{\He}[0]{\mathbf{H}}
\newcommand{\comp}[0]{\circ}
\newcommand{\parcial}[2]{\frac{\partial #1}{\partial #2}}
\newcommand{\eps}[0]{\varepsilon}
\newcommand{\lthen}[0]{\Rightarrow}
\newcommand{\liff}[0]{\Leftrightarrow}
\newcommand{\piso}[1]{\left\lfloor{x}\right\rfloor}
\newcommand{\error}[0]{{\cal E}}
\newcommand{\Ode}[0]{{\cal O}}

\newcommand{\caja}[1]{\noindent \fbox{\parbox{\textwidth}{#1}}}


\newtheorem{teo}{Teorema}
\newtheorem{lema}{Lema}
\newtheorem{coro}{Corolario}
\newtheorem{defi}{Definici\'on}
\newtheorem{obs}{Observaci\'on}

% proof
\newenvironment{demo}{{\noindent \textbf{Demo: }}}{\hfill\rule{2mm}{2mm}\par}

\newenvironment{items}{
		\vspace*{-2\topsep}
		\begin{itemize} 
		\addtolength{\itemsep}{-0.5\baselineskip}
		}{\end{itemize}\vspace*{-\topsep}}

\lhead{Metodos Num\'ericos}
\rhead{Apunte de repaso general}

\cfoot{$\thepage$ de \pageref{theend}}

\begin{document}
Disclaimer: Este apunte no es autocontenido y fue pensado como un repaso 
de los conceptos, no para aprenderlos de aqu'i directamente.

%\begin{multicols}{2}\tableofcontents\end{multicols}

\section{Aritm\'etica de la computadora}

\textbf{Representaci'on en punto flotante}. Los n'umeros en punto flotante
se representan como $s x b^y$ donde $b$ es una constante dada, $s$ es un bit
de signo, $x$ es llamada mantisa y esta entre 0 y $b-1$ y $y$ es llamado
exponente y es un entero. En la computadora usual se usa $b=2$ y la mantisa se
representa solo la parte a partir del segundo d'igito decimal, ya que por
convenci'on la mantisa empieza ``0.1...''. Esto se llama normalizaci'on.
Esta representaci'on implica que la distribuci'on de los n'umeros no es
del todo lineal, pero tampoco del todo logar'itmica, sino un h'ibrido entre
ambas (i.e., entre cada par de potencias de 2 consecutivas hay la misma
cantidad de n'umeros distribuidos uniformemente).

\textbf{Errores}. El error absoluto de $A$ representado como $a$ se define
como $|A-a|$ y el error relativo como $|A-a|/|A|$. \\
El error del c'alculo de una funci'on $f(x_1,...,x_n)$ se define como
$$\error(f(x_1,...,x_n)) = 
	\error_f + \sum_{i=1}^n \frac{\parcial{f}{x_i}}{f} x_i \error(x_i),$$
donde $\error_f$ es el error intr'inseco de hacer la operaci'on $f$ y 
$\error(x_i)$ es el error de representaci'on de $x_i$.

\textbf{Condici'on y estabilidad}. \\
N'umero de condici'on: t'erminos que acompa'nan a los errores en las variables.
(est'a bien condicionado si los n'umeros de condici'on son acotables por
una constante). \\
Estabilidad del algoritmo: t'erminos que acompa'nan a los errores de las
operaciones (es estable si la estabilidad es acotable por una constante). \\
Para dos algoritmos distintos que calculan la misma expresi'on aritm'etica
los n'umeros de condici'on son iguales. La estabilidad no necesariamente.

\vspace{0.8cm}

\caja{\textbf{NOTA:}
Para las pr'oximas secciones usaremos la siguiente notaci'on: Los problemas
lineales se identifican con una matriz $A$ de $n \times n$ y un vector $b$ de 
$n \times 1$. El problema entonces es buscar un vector $x$ de $n \times 1$ tal
que $Ax = b$.}

\section{Factorizaci\'on LU}

La factorizaci'on LU sirve para resolver varios problemas lineales. Si el
problema est'a dado por $Ax = b$, la idea es encontrar matrices 
$L$ triangular inferior y $U$ triangular superior tales que 
$LU = A$, de manera de resolver el sistema $LUx = b$. Esto 'ultimo
es facil de hacer
en 2 pasos: buscamos $x_0$ tal que $Lx_0 = b$, lo cual se puede hacer f'acil
porque es un sistema lineal en el que la matriz es triangular. Luego buscamos
$x$ tal que $Ux = x_0$, que es f'acil por la misma raz'on. De este modo, 
$LUx = Lx_0 = b$, y el $x$ es efectivamente la soluci'on del sistema original. 
La idea de la factorizaci'on es que puede obtenerse en el mismo tiempo
necesario para resolver un sistema lineal $\Ode(n^3)$, pero una vez que se
tiene resolver un problema $Ax = b_i$ para cualquier $b_i$ puede hacerse en
$\Ode(n^2)$ como ya fue descripto. Esto quiere decir que si uno tiene que
resolver $k$ problemas $Ax = b_i (1 \leq i \leq k)$ haciendo $k$
eliminaciones gaussianas tarda $\Ode(n^3k)$, pero haciendo la factorizaci'on
$LU$ de $A$ tarda $\Ode(n^3 + n^2k)$ lo cual es siempre mejor o igual (en
particular notemos que si $k=1$, no estamos perdiendo nada de complejidad por
hacer la factorizaci'on en lugar de la eliminaci'on gaussiana directamente).

Para desambiguar a veces se pide que la diagonal de $L$ sean solo unos 
(factorizaci'on de Doolittle) o la diagonal de $U$ solo unos (factorizaci'on
de Crout).

\textbf{Eliminaci'on Gaussiana}. La eliminaci'on gaussiana es un m'etodo para
resolver problemas lineales. En particular tambi'en sirve para producir una
factorizaci'on $LU$. Aqu'i describiremos la eliminaci'on utilizada para hacer
dicha factorizaci'on, asumiendo que no es necesario pivoteo. En particular
produciremos la factorizaci'on de Doolittle (es decir, $L$ con unos en la
diagonal). \\
La idea de la eliminaci'on es encontrar dos sucesiones de matrices
$L^{(k)}$ y $U^{(k)}$ tal que para todo $L^{(n)}U^{(n)}=A$ y $L^{(n)}$
es triangular inferior y $U^{(n)}$ triangular superior. Como invariante
adicional tomaremos que para todo $k$ $L^{(k)}U^{(k)}=A$, y para $i < j < k$
se cumple $U^{(k)}_{i,j} = 0$, es decir, $U^{(k)}$ tiene las primeras $k$
columnas sin $0$s abajo de la diagonal, y $L^{(k)}$ es siempre triangular
inferior con unos en la diagonal. Notemos que si $k=n$ el invariante implica 
la postcondici'on. Las sucesiones de matrices se construyen seg'un el siguiente
algoritmo: \\
\caja{Sean $L^{(1)} = I$ y $U^{(1)} = A$ \\
para $k = 2$ hasta $n$ \\
\tab \textit{aqu'i asumiremos que $U^{(k-1)}_{k-1,k-1} \neq 0$ (no es
	necesario pivoteo)} \\
\tab Sea las matrices $M$ y $M^{-1}$ de $n \times n$ e inversas entre s'i dadas
por:
$$M_{i,j} = \begin{cases}
	1 & i = j \\
	-U^{(k-1)}_{i,k} / U^{(k-1)}_{k,k} &  i > k \mbox{ y } j = k\\
	0 & \mbox{ en otro caso}
\end{cases} 
\hspace{1cm}
M^{-1}_{i,j} = \begin{cases}
	1 & i = j \\
	U^{(k-1)}_{i,k} / U^{(k-1)}_{k,k} &  i > k \mbox{ y } j = k\\
	0 & \mbox{ en otro caso}
\end{cases}$$
\tab Sean $L^{(k)} = L^{(k-1)}M^{-1}$ y $U^{(k)} = MU^{(k-1)}$}
Notemos que $L^{(1)}U^{(1)} = A$ trivialmente y que $L^{(k)}U^{(k)} = 
L^{(k-1)}M^{-1}MU^{(k-1)} = L^{(k-1)}U^{(k-1)} = A$ por lo cual el invariante
se conserva. El hecho de que $L^{(k)}$ es siempre triangular superior y 
$U^{(k)}$ lo es en sus primeras $k$ columnas sale de observar la forma que toma
la multiplicaci'on por $M^{-1}$ y $M$ respectivamente (en el caso de $L$ es
producto de dos matrices triangular inferior, en el caso de $U$ se ve como
la multiplicaci'on solo cambia filas por debajo de la $k-'esima$ y siempre
manteniendo las columnas menores a $k$ intactas y anulando la columna $k$). \\
Es importante notar que las matrices $L^{(k)}$ y $U^{(k)}$ pueden ir
guardandose todas en el mismo espacio de $n \times n$. A su vez, la matriz $M$
puede guardarse utilizando solo $\Ode(n)$ espacio y los productos de $L$ o $U$
por $M$ o $M^{-1}$ realizados consiguientemente en $\Ode(n^2)$. De este modo,
todo el proceso utiliza solo $n^2 + \Ode(n)$ memoria y se realiza en
$\Ode(n^3)$ tiempo.

\textbf{Factorizaci'on $PA=LU$}. Si la eliminaci'on Gaussiana precisara de
pivoteo, y lo conocemos de antemano (o lo calculamos), podemos factorizar
$PA=LU$ con $P$ la matriz de permutaci'on que hace los intercambios de fila
necesarios para pivotear la eliminaci'on Gaussiana. Esto tambi'en conduce a
una eficiente resoluci'on de los sistemas $Ax= b_i$, ya que $P$, por ser de
permutaci'on, es trivialmente inversible.

\textbf{Propiedad de existencia}. Si las $n$ submatrices principales de $A$
son no singulares, entonces $A$ tiene factorizaci'on $LU$.

\section{Factorizaci\'on QR}

Una matriz $Q$ de $n \times n$ es ortogonal si todas sus columnas tienen
norma $1$ y el producto interno de 2 distintas de ellas es $0$. En particular
cumplen $QQ^t = Q^tQ = I$ y por lo tanto se cumple lo mismo para las filas.

La factorizaci'on $QR$ de una matriz consiste en, dada $A$, encontrar $Q$
ortogonal y $R$ triangular superior tal que $QR = A$. De esta manera, para
resolver el sistema $Ax = b$ debemos resolver $QRx = b$ o equivalentemente
$Rx = Q^{-1}b = Q^t b$, cosa que podemos hacer f'acilmente (si es posible) 
ya que $R$ es triangular. Para todo $A$ existe la factorizaci'on $QR$.

La factorizaci'on es conveniente para resolver varios sistemas de la forma
$Ax = b_i$ para varios $b_i$, ya que solo hace falta computar la factorizaci'on
una vez (porque $A$ est'a fijo) y luego se puede resolver cada sistema en 
$\Ode(n^2)$ en lugar del $\Ode(n^3)$ para resolver un sistema en general.

Para los m'etodos de obtener una factorizaci'on $QR$ es importante notar que
el producto de matrices ortogonales es ortogonal.

\textbf{Transformaci'on de Givens}. La idea es ir multiplicando $A$
sucesivamente por matrices de rotaci'on (que son ortogonales) hasta que se
convierta en $R$. El producto de las inversas de las rotaciones es, por lo
tanto, $Q$. Para que $Q^t$ sea una rotaci'on debe ser, para ciertos $p,q$:
$$Q^t_{i,j} = \begin{cases}
1 & i = j \notin \{p,q\} \\
0 & i \neq j \land \{i,j\} \neq \{p,q\} \\
\cos \theta & i = j \in \{p,q\} \\
\sen \theta & i = p, j = q \\
-\sen \theta & i = q, j = p \\
\end{cases}
\hspace{0.5cm}
\mbox{por ejemplo, para } p=2 \mbox{ y } q=5 \mbox{:}
\left(
\begin{array}{ccccc}
1 & 0 & 0 & 0 & 0\\
0 & \cos \theta & 0 & 0 & \sen \theta \\
0 & 0 & 1 & 0 & 0 \\
0 & 0 & 0 & 1 & 0 \\
0 & -\sen \theta & 0 & 0 & \cos \theta \\
\end{array}
\right)
$$
En particular queremos $\theta$ tal que provoque un cero particular en la
matriz $A$. Miremos por ejemplo el caso de $2 \times 2$:
$$\left(
\begin{array}{cc}
\cos \theta & \sen \theta \\
-\sen \theta & \cos \theta \\
\end{array}\right)
\left(\begin{array}{l}
x_1 \\ x_2
\end{array}\right)
=
\left(\begin{array}{l}
* \\ 0
\end{array}\right)
\lthen
\begin{cases}
\begin{array}{rcl}
\cos \theta x_1 + \sen \theta x_2 &=& * \\
-\sen \theta x_1 + \cos \theta x_2 &=& 0 \\
\end{array}
\end{cases}
$$
y luego tomo como soluci'on $\sen \theta = x_2 / \norma{(x_1,x_2)}$ y $
\cos \theta = x_1 / \norma{(x_1,x_2)}$ resulta que 
$$Q^t = \frac{1}{\norma{(x_1,x_2)}} \left(\begin{array}{cc}
x_1 & x_2 \\
-x_2 & x_1 \\
\end{array}\right)$$
produce el cero que quer'iamos. Podemos generalizar esto a $Q$ de $n \times n$
reemplazando $\sen \theta = A_{q,p}$ y $\cos \theta = A_{p,p}$ en la
definici'on original de $Q^t$. Haciendo eso iterativamente para cada $p,q$
donde precisemos un cero nos da la factorizaci'on:
\caja{
Sea $Q^* = I$ \\
Sea $R = A$ \\
para $q = 1$ hasta $n$ \\
\tab para $p = q+1$ hasta $n$ \\
\tab\tab Sea $Q^t$ como fue definido arriba para $p,q$. \\
\tab\tab $Q^* = Q^* Q$ \\
\tab\tab $R = Q^t R$
}
El invariante es que $Q^*R = A$, que se mantiene porque en cada paso
multiplico en medio de ambos por $QQ^t = I$. $Q^*$ es siempre ortogonal por
ser producto de ortogonales y $R$ mantiene ceros en las posiciones $p,q$ por
las que ya se pas'o. Al final del algoritmo, $R$ tiene ceros en todas las
posiciones debajo de la diagonal. Dado que las matrices $Q$ y $Q^t$ se pueden
tienen solo 4 elementos distintos a la identidad y que
los productos pueden hacerse por lo tanto en $\Ode(n)$ el tiempo total del
algoritmo de factorizaci'on es $\Ode(n^3)$.

\textbf{Transformaci'on de Householder}. La idea es usar reflexiones respecto
de rectas que pasan por el $0$, que tambi'en pueden verse como productos de
matrices ortogonales. En particular, si queremos reflejar por la recta $v$:
sea $u$ perpendicular a $v$ ($u^tv = 0$) de norma $1$. Si la reflexi'on est'a
dada por la matriz $Q$ queremos: $Qu = -u$ y $Qv = v$. Observemos que tomando
$Q = I - 2uu^t$ esto se cumple ya que $Qu = (I-2uu^t)u = u - 2u(u^tu) = 
u - 2u \norma{u} = -u$ y $Qv = (I-2uu^t)v = v - 2u(u^tv) = v$. \\
La idea para factorizar es generar ceros en una columna por vez, generando
matrices $Q^t_p$ que al multiplicarlas por $A$ dejan ceros en la columna $p$,
dejando intactas las columnas anteriores a $p$. De esta manera, 
$R = \prod_{p=n}^1 Q^t_p A$ y $Q = \prod_{p=1}^n Q_p$. Es claro que $QR = A$.
Como utilizamos reflexiones, generar ceros en una columna debe modificar alg'un
otro elemento, para mantener fija la norma de la columna. En particular,
modificaremos la diagonal. Sea $x_1,...,x_n$ la columna $p$ a modificar.
Queremos como resultado $y_i=x_i$ si $i<p$ y $y_i=0$ si $i>p$, por lo cual
resulta $y_p = \norma{(x_p,x_{p+1},...,x_n)}$. Ahora que sabemos que queremos
mandar el punto $x$ al $y$, solo debemos escoger la recta $v$ y luego una
perpendicular $u$ de norma $1$. La recta que une $x$ con $y$, $x-y$ es
perpendicular al eje de reflexi'on por su definici'on, por lo tanto nos sirve.
Para que sea de norma $1$, simplemente lo normalizamos y usamos 
$u_p=(x-y)/\norma{x-y}$. Definamos entonces $Q^t_p$ seg'un los siguientes
bloques:
$$Q^t_p = \left(\begin{array}{cc}
I_{p-1 \times p-1} & 0 \\
0 & I_{n-p+1 \times n-p+1} - 2u_p u_p^t
\end{array}\right)$$
donde $u_p$ es como fue descripto anteriormente.
Se ve que las matrices $Q^t_p$ y $Q_p$ pueden ser representada simplemente 
como el vector $u_p$ de tama~no $n$ y por lo tanto podemos resolver sus
productos por otras matrices en $\Ode(n^2)$. Dado que tenemos $\Ode(n)$ de
esos productos, la complejidad total resultante para hayar la factorizaci'on
es de $\Ode(n^3)$.

\textbf{Propiedad de existencia y unicidad}. Si $A$ es no singular entonces
existen 'unicas matrices $Q$ ortogonal y $R$ triangular superior tal que
la diagonal de $R$ consta solo de elementos estrictamente positivos.

\section{Resoluci\'on de sistemas con matrices especiales}

\textbf{Matriz estrictamente diagonal dominante}. Llamamos as'i a una matriz
$A$ cuando $2A_{i,i} > \sum_{j=1}^n |A_{i,j}|$. En estos casos, $A$ es no
singular y no es necesario pivoteo al hacer Gauss, por lo cual tiene
factorizaci'on $LU$.

\textbf{Matrices banda}. Llamamos matriz banda $pq$ a $A$ cuando $A_{i,j} = 0$
siempre que $i+p \leq j$ y $j+q \leq i$. En particular, si $A$ tiene
factorizaci'on $LU$, $L$ tiene banda $q$, $U$ tiene banda $p$ y la eliminaci'on
gaussiana para encontrar la factorizaci'on se puede hacer en $\Ode(npq)$.

\textbf{Matrices sim'etricas}. Una matriz $A$ es sim'etrica si $A = A^t$. En
este caso la factorizaci'on $LU$ de $A$ (si es que existe) puede escribirse
en t'erminos de $L$ y una matriz diagonal $A = LU = LD(L^t)$. Si aparte la
matriz es definida positiva podemos estar seguros que tiene factorizaci'on
$LU$ y aparte la podemos descomponer como $L'(L')^t$ tomando $L' = LD^{1/2}$.

\section{Inestabilidad num\'erica al resolver sistemas lineales}

Al resolver sistemas lineales se pueden presentar problemas num'ericos. Estos
a veces pueden ser resueltos o minimizados mejorando la forma de resolverlo.

\textbf{Eliminaci'on Gaussiana}. Durante la eliminaci'on gaussiana a veces hay
``necesidad'' de pivotear por encontrar un $0$ en la diagonal. Dado que
podemos pivotear, a veces puede convenirnos num'ericamente hacerlo, mas all'a
de que no sea matem'aticamente necesario. En particular, intentaremos que el
n'umero sobre la diagonal (por el que vamos a dividir) sea lo mas grande
posible en valor absoluto, ya que sabemos que dividir por n'umeros cercanos a
cero es problem'atico num'ericamente. Denominamos \emph{pivoteo parcial} a 
buscar el mejor coeficiente dentro de la columna correspondiente. Denominamos
\emph{pivoteo total} a buscarlo en toda la submatriz que queda a la derecha
y abajo del coeficiente actual (necesitando un intercambio de filas y uno de
columnas).

\textbf{N'umeros de condici'on}. Denominamos el n'umero de condici'on de una
matriz $A$ a $K(A) = \norma{A} \norma{A^{-1}} \geq 1$. Este n'umero esta
relacionado con cu'an problem'atico num'ericamente es resolver un sistema con
matriz $A$: mientras mas grande sea, mas problem'atico. Esto se ilustra con
la siguiente propiedad: \\
Propiedad: Sea $\bar{x}$ soluci'on aproximada de $Ax = b$ con $A$ no singular
y residuo $r = b - A\bar{x}$. Podemos acotar el error absoluto y relativo de
la soluci'on de la siguiente manera:
\begin{items}
\item $\norma{x-\bar{x}} \leq \norma{r} \norma{A^{-1}}$
\item $\norma{x-\bar{x}} / \norma{x} \leq K(A) \norma{r} / \norma{b}$
\end{items}

\section{M\'etodos iterativos para sistemas lineales}

Los m'etodos iterativos para solucionar sistemas lineales generan una sucesi'on
de soluciones aproximadas $x_n$ que bajo ciertas condiciones converge a la
soluci'on del sistema.

\textbf{Autovalores y radio espectral}. Un autovalor $\lambda$ y su
correspondiente autovector $v \neq 0$ de una matriz $A$ son tales que 
$Av=\lambda v$. Los autovalores tambi'en son las raizes del polinomio
$\det(A-\lambda I)$. \\
El radio espectral $\rho(A)$ se define como el m'aximo valor absoluto de un
autovalor de la matriz $A$. \\
$A$ es una matriz convergente si $\limite{k}{\infty} A^k = 0$. \\
Propiedad: $\rho(A) \leq \norma{A}$ para toda norma inducida. \\
Propiedad: $A$ es matriz convergente $\liff \rho(A) < 1$. \\
Propiedad: Si $\norma{A} < 1$ entonces $I-A$ es no singular y 
$(I-A)^{-1} = \sum_{k=0}^\infty A^k$.

\textbf{M'etodos iterativos en general}. La idea es utilizar la iteraci'on
$x^{(k+1)} = T x^{(k)} + c$ para hallar la soluci'on del sistema
$x = T x + c$. \\
Propiedad: Dicha sucesi'on converge $\liff \rho(T) < 1$. \\
Propiedad: Si para una norma inducida $\norma{T} < 1$ entonces
$x^{(k+1)} = T x^{(k)} + c$ converge y
$\norma{x-x^k} \leq \norma{T}^k \norma{x^{(0)}-x}$.

\textbf{Jacobi}. Dentro del esquema general, usaremos 
$T = D^{-1}(L+U)$ y $c = D^{-1}b$ donde $A = D - L - U$, $D$ es diagonal y $L$
y $U$ son triangular inferior y superior respectivamente con ceros en la
diagonal. La idea sale de: 
$Ax = b \liff (D-L-U)x = b \liff x = D^{-1}b + D^{-1}(L+U)x$.

\textbf{Gauss Siedel}. Sea $A = D - L - U$ igual que antes. Usaremos 
$T = (D-L)^{-1}U$ y $c = (D-L)^{-1}b$. Esto sale de $Ax = b \liff (D-L-U)x = b
\liff x = (D-L)^{-1}Ux + (D-L)^{-1}b$.

\textbf{Convergencias conocidas}. Si $A$ es estrictamente diagonal dominante,
entonces Jacobi y Gauss Siedel convergen.

\subsection{Direcciones conjugadas}

\textbf{Direcciones conjugadas}. $n$ vectores $u_1$,...,$u_n$ se llaman 
$A$-conjugados sii $\forall i \neq j\ u_i^t A u_j = 0$. Si adem'as
$\forall i\ u_i^t A u_i = 1$ se llaman $A$-ortogonales.

Supongamos que tenemos que resolver $Ax=b$ donde $A$ es sim'etrica y definida
positiva. Sea $Q(x) = x^t A x - 2 x^t b$. Se ve que $\D Q(x) = 0$ y 
$\He Q(x) = 2A$ es definido positivo, por lo cual $x$ es m'inimo de $Q$,
entonces resolveremos el sistema buscando dicho m'inimo.

Para esto utilizaremos una iteraci'on de la forma 
$x_k = x_{k-1} + \alpha_{k-1} d_{k-1}$, donde $\alpha$ indica cu'anto me muevo
y $d$ indica la direcci'on del movimiento. Si tengo dada la direcci'on, puedo
plantear cuanto moverme buscando que $Q$ se minimice lo m'aximo posible. Como
esto es una ecuaci'on en una variable, simplemente busco el cero de la
derivada respecto de $\alpha$, que al resolverlo queda
$\alpha_{k-1} = d_{k-1}^t(b-Ax_{k-1}) / d_{k-1}^tAd_{k-1}$. \\
Propiedad: Si $d_0,...,d_{n-1}$ son $A$-ortogonales y eligiendo $\alpha$
seg'un se explic'o, definiendo $x_i = x_{i-1} + d_{i-1} \alpha_{i-1}$ resulta 
$Ax_n = b$. \\
\note{dibujito de direcciones conjugadas} \\
La idea es irse moviendo por direcciones conjugadas, que son linealmente
independientes. Se puede ver como moverse una vez en cada ``coordenada''. \\
Propiedad: Si elegimos $d_k = r_k + \beta_k d_{k-1}$ donde $r_k = b-Ax_k$ y
$\beta_k = r_k^t A d_{k-1} / ( d_{k-1}^t A d_{k-1} )$ entonces las direcciones
resultan $A$-ortogonales.

\section{Sistemas no lineales}

Los sistemas no lineales pueden verse como buscar ceros de funciones de varias
variables. En particular, sea $F: \R^n \to \R^n$ (asumiendo que tenemos $n$
variables y $n$ ecuaciones) e intentemos buscar sus ceros. Sea $x^*$ tal que
$F(x^*)=0$.

\textbf{Newton}. Podemos generalizar el m'etodo de Newton utilizando la
ecuaci'on $F(x_k) + J(x_k)(x - x_k) \sim F(x)$, lo cual deriva en la 
iteraci'on $x^{(k+1)} = x^{(k)} - J^{-1}(x^{(k)})F(x_k)$ donde $J$ es el
jacobiano o la matriz diferencial de $F$. Multiplicar por la inversa de dicha
matriz es el an'alogo a dividir por la derivada de Newton para 1 variable. \\
Propiedad: Si $J(x^*)$ es no singular, $\norma{J^{-1}(x^*)} \leq \beta$, $J$
es Lipschitz $F$ es $C^1$ en un intervalo alrededor de $x^*$, entonces
el m'etodo converge cuadr'aticamente si $x_0$ est'a suficientemente cerca de
$x$.

\textbf{Criterio de Broyden}. Broyden puede verse como una generalizaci'on de
secante a muchas dimensiones o tambi'en como una simplificaci'on de Newton, que
en lugar de utilizar el Jacobiano, lo aproxima seg'un la 'ultima diferencia
finita que tiene (es decir, toma $J(x_n)(x - x_n) \sim F(x) - F(x_{n-1})$).
El problema es que, al contrario que en una dimensi'on, esto resulta en un
sistema subdeterminado. Lo que Broyden propone es tomar la soluci'on que
minimiza los cambios a la aproximaci'on del Jacobiano, de manera de obtener un
m'etodo mas eficiente.
\note{me queda la duda de si est'a bien as'i o hay que agregarle formalismo 
(cuentitas)}

\section{Ceros de funciones}

El problema es encontrar los $x^*$ tal que $f(x^*) = 0$ para una funci'on dada
$f$. Hay m'etodos particulares varias familias particulares de funciones, pero 
nos concentraremos en m'etodos ``generales'', para familias grandes.

Un algoritmo que busca ceros de funciones genera una sucesi'on de resultados
$x_n$ de manera que $\limite{n}{\infty} x_n = x^*$.

\textbf{Convergencia $p$}. La sucesi'on $x_n$ tiene convergencia $p$ si existe
un $k$ tal que:
$$\limite{n}{\infty} \ell_{n+1} / \ell_n = c \neq 0 \mbox{ donde }
	\ell_n = |x_n - x^*| \mbox{, y } \ell_{n+1} \leq k\ {\ell_n} ^ p.$$
Si $p=1$ decimos convergencia lineal, si $p=2$, cuadr'atica.
Si $\limite{n}{\infty} \ell_{n+1} / \ell_n = 0$ la convergencia
es superlineal (mejor que lineal pero peor que cuadr'atica).

\textbf{Criterios de parada}.
\begin{items}
\item Error absoluto $|x_n - x_{n-1}| < \eps$.
\item Error relativo $|x_n - x_{n-1}| / |x_n| < \eps$.
\item Error sobre la funci'on $|f(x_n)| < \eps$.
\item Sobre diferencia de funciones $|f(x_n) - f(x_{n-1})| < \eps$.
\item Sobre diferencia de funciones con error relativo 
	$|f(x_n) - f(x_{n-1})| / |f(x_n)| < \eps$.
\end{items}

\textbf{Bisecci'on}. Se basa en el teorema de Bolzano: Dada $f : [a,b] \to \R,
f(a)f(b) < 0, \exists c \in (a,b)\mbox{ tq }f(c) = 0$. Algoritmo: \\
\caja{Sean $a$ y $b$ tal que $f(a)f(b) < 0$. \\
Mientras se desee mas precisi'on \\
\tab Si $f((a+b)/2)f(b) > 0$ \\
\tab\tab $a = (a+b)/2$ \\
\tab sino \\
\tab\tab $b = (a+b)/2$. \\
Devolver $(a+b)/2$}
Propiedad: La convergencia del m'etodo de bisecci'on es lineal.

\textbf{Punto Fijo}. Decimos punto fijo de una funci'on $g$ al punto $x^*$ tal
que $g(x^*) = x^*$. Si queremos buscar los ceros de $f$ podemos definir 
$g(x) = f(x) + x$ de manera que los puntos fijos de $g$ son los ceros de $f$.
Tambi'en podemos hacer $g(x) = $ cualquier forma de despejar $x$ en la 
expresi'on $f(x) = 0$. \\
Propiedad: Si $g : [a,b] \to [a,b]$ es continua entonces tiene un punto fijo.
Importante notar que la imagen de $[a,b]$ tiene que ser $[a,b]$ cerrado!.
Si adem'as $|g'(x)| \leq K < 1$ para todo $a \leq x \leq b$ entonces el punto
fijo es 'unico. \\
Algoritmo: Ir haciendo $x_n = g(x_{n-1})$. \\
Propiedad: Si $g : [a,b] \to [a,b]$ es continua y $|g'(x)| \leq K < 1$
entonces el algoritmo converge al 'unico punto fijo. \\
Propiedad: Si $g$ es como en la propiedad anterior y 
$g'(p) = ... = g^{(n-1)}(p) = 0$ entonces la convergencia del algoritmo es por 
lo menos $n$.
\note{agregar grafiquito de punto fijo}

\textbf{Algoritmo de Newton}. 
En particular veamos el punto fijo de $g(x) = x-h(x)f(x)$. Si $g(x^*) = x^*$
entonces $h(x^*)f(x^*)=0$ y si $h(x^*) \neq 0$, entonces $x^*$ es cero de $f$.
En particular, $g'(x) = 1-h'(x)f(x)-h(x)f'(x)$. Como queremos una derivada
igual a $0$ (para asegurar convergencia cuadr'atica) precisar'iamos
$0 = g'(x^*) = 1-h(x^*)f'(x^*)$, entonces tomemos $h(x) = 1/f'(x) \lthen 
g(x) = x-f(x)/f'(x)$. \\
Propiedad: Sea $f \in C^2 [a,b], x^* \in [a,b], f(x^*) = 0$ y $f'(x^*) \neq
0$. Existe un $\delta>0$ tal que la sucesi'on $x_{n+1} = x_n - f(x_n)/f'(x_n)$ 
converge a $x^*$ si $x_0 \in (x^* - \delta, x^* + \delta)$. \\
Propiedad: Si $f$ adem'as es creciente y convexa, entonces tiene un 'unico
cero y el algoritmo de newton converge a el para todo $x_0$. \\
Tambi'en podemos interpretar Newton desde el polinomio de Taylor, ya que
$0 = f(x) \sim f(x_0) + f'(x_0)(x - x_0) \lthen x = f(x_0) / f'(x_0) + x_0$. \\
Observaci'on: Si hay ceros m'ultiples la convergencia cuadr'atica puede
perderse.

\textbf{M'etodo de la secante}.
La idea es aproximar la derivada de Newton por una diferencia finita dividida.
Tambi'en puede verse como tomar la recta secante a los 'ultimos dos puntos
encontrados (derivada estimada) y utilizar como nuevo punto su intersecci'on
con la funci'on.
$$x_{n+1} = x_n - f(x_n) \frac{x_n - x_{n-1}}{f(x_n) - f(x_{n-1})}$$
La convergencia de este m'etodo es $\varphi = (1+\sqrt{5})/2$ si se eligen los
puntos iniciales suficientemente cerca del buscado.

\textbf{M'etodo Regula Falsi}. Es como bisecci'on pero en lugar de elegir el
punto siguiente como el promedio de los lados $c = (a+b)/2$, elige la
intersecci'on de la recta secante que une $(a,f(a))$ con $(b,f(b))$. 
Formalmente $c = (f(b)a - f(a)b)/(f(b)-f(a))$. El m'etodo recibe su nombre del
hecho de que no necesariamente converge bien, aunque lo pareciera.

\section{Interpolaci\'on}

La interpolaci'on es, dados pares de valores $(x_1,y_1),...,(x_n,y_n)$ 
(mediciones), encontrar una funci'on $f$ tal que $\forall i\ f(x_i) = y_i$. En
particular la interpolaci'on polin'omica es cuando las funciones $f$ que
podemos elegir se restringen a polinomios.

\textbf{Polinomio de Lagrange}. El Polinomio de Lagrange es el 'unico polinomio 
de grado m'inimo que interpola un conjunto de puntos $(x_i,y_i)$. Est'a
definido por
$$p(x)=\sum_i \left( y_i \prod_{j \neq i} \frac{x - x_j}{x_i - x_j} \right).$$
Propiedad: Si todo $x_i \in [a,b]$ y $f$ es $C^{n+1}$ entonces para todo $x$ 
existe $\xi_x \in (a,b)$ tal que 
$f(x) = p(x) + f^{(n+1)}(\xi_x) \prod_i (x-x_i) / (n+1)!$. 

\textbf{Diferencias divididas}. Es una forma recursiva de escribir el
polinomio de lagrange. Sirve para hacerlo incrementalmente. Supongamos que
el polinomio es $p(x) = \sum_{i=0}^n a_i \prod_{j < i+1} (x - x_j)$. La idea es
definir
$f[x_i x_{i+1} ... x_{i+k}] = (f[x_{i+1} ... x_{i+k}] - f[x_i ... x_{i+k-1}])
	/ (x_{i+k} - x_i)$ y $f[x_i] = f(x_i)$. Se puede ver que el coeficiente
$a_k$ debe ser $f[x_1 ... x_{k-1}]$.

\textbf{Splines}. La idea de un spline es interpolar con polinomios, pero 
'estos pueden ser distintos para cada pareja de mediciones consecutivas
(asumimos que $x_1 < x_2 < ... < x_n$. Buscaremos entonces polinomios
$p_1,...,p_{n-1}$ tal que para todo $i$ se cumpla $p_i(x_i) = y_i$ y 
$p_i(x_{i+1}) = y_{i+1}$ (1). Si los polinomios son de grado $1$ (rectas) 
estas ecuaciones completamente los definen. Si vamos a grados mas grandes,
necesitamos mas ecuaciones para definirlos. En particular, para que la curva
sea suave podemos pedir $p_i'(x_{i+1}) = p_{i+1}'(x_{i+1})$ (2) o lo mismo
para la derivada segunda, tercera, etc'etera. En particular consideremos los
splines c'ubicos (con polinomios de grado 3). Cada polinomio introduce $4$ 
variables (sus 4 coeficientes). Tenemos $2n-2$ ecuaciones de tipo (1) y 
$n-2$ ecuaciones de tipo (2). Si igualamos en los puntos intermedios tanto
la derivada primera como la segunda, esto nos da $2n-2 + n-2 + n-2$ = $4n-6$
ecuaciones para nuestras $4n-4$ variables ($4$ para cada uno de los $n-1$
polinomios). Las 2 ecuaciones que hacen falta se completan seg'un el tipo de
spline. Natural: Las derivadas segundas en los extremos son cero.
Sujeto: La derivada primera en cada extremo se iguala a la derivada de la
funci'on original. Se puede demostrar en ambos casos que el sistema planteado
tiene siempre una 'unica soluci'on.

\section{Cuadrados m'inimos}

El problema de cuadrados m'inimos es encontrar, dadas mediciones 
$(x_1,y_1),...,(x_n,y_n)$ una funci'on $f$ tal que minimice el error
cuadr'atico medio $\sum_i (f(x_i)-y_i)^2$.

\textbf{Cuadrados m'inimos lineales}. El problema de cuadrados m'inimos
lineales es cuando restringimos la funci'on $f$ a una funci'on lineal. Puedo
encontrar $f$ planteado minimizar $\sum_i (a x_i + b - y_i)^2$ como funci'on
de $a$ y $b$ e igualando el gradiente de dicha expresi'on a $0$.

\textbf{Cuadrados m'inimos polinomiales}. Aqu'i restringimos $f$ a los
polinomios de cierto grado m'aximo. Tambi'en se resuelve mediante un sistema
lineal, que resulta escribir la expresi'on a minimizar en funci'on de los
coeficientes de $f$ y luego igualar el gradiente a $0$.

Ahora generalicemos el problema a muchas variables: Tenemos una tabla de
valores 
$$(x_1^{(1)},x_2^{(1)},...,x_k^{(1)},y^{(1)}),...,
(x_1^{(n)},x_2^{(n)},...,x_k^{(n)},y^{(n)})$$
 y queremos la funci'on $f$ que
minimiza $\sum_i (f(x_1^{(i)},...,x_k^{(i)})-y_i)^2$.

\textbf{Cuadrados m'inimos lineales en varias variables}. Si restringimos $f$
a funciones lineales, podemos ver que el problema se trata de encontrar un
$x$ que minimiza $\norma{Ax-b}^2$ donde $A$ es la matriz de $n \times k$ donde 
$A_{i,j} = x_j^{(i)}$ y $b_i = y_i$. \\
Propiedad: El problema de cuadrados m'inimos lineales siempre tiene soluci'on.
La soluci'on es 'unica $\liff$ no existe un vector $v$ no nulo tal que 
$Av = 0$. \\
Propiedad: Si $x$ es la soluci'on de cuadrados m'inimos lineales, entonces
$A^t A x = A^t b$. \\
Esta 'ultima propiedad da una forma de resolver el problema, resolviendo el
sistema $A^t A x = A^t b$ utilizando alg'un m'etodo ya visto ($LU$, $QR$, 
etc). \\
Observaci'on: $A^t A$ es sim'etrica y semi definida positiva, lo cual trae
problemas num'ericos al resolver el sistema.

\section{C'alculo de autovalores}

Propiedad: Si A y B son similares (i.e., existe $S$ tq $S^{-1}BS = B$) tienen
los mismos autovalores. \\
Teorema: Sea $R_i = \{z |\ |z-A_{i,i}| \leq \sum_{j \neq i} A_{i,j}\}$. Si
$\left(\bigcup_{i \in K} R_i\right) \cap \left(\bigcup_{i \notin K} R_i\right)
	= \emptyset \lthen$
en $\bigcup_{i \in K} R_i$ hay $|K|$ autovalores de $A$.

\textbf{M'etodo de la potencia}. Si tengo una base de autovectores 
$<v_1,...,v_n>$ para todo $x$ existen $\beta_1,...,\beta_n$ tal que
$x = \sum_i \beta_i v_i$. Fijo un $x$. \\
\begin{eqnarray*}
x &=& \sum \beta_i v_i \\
Ax &=& \sum \beta_i A v_i = \sum \lambda_i \beta_i v_i \\
A^kx &=& \sum \lambda_i^k \beta_i v_i = 
	\lambda_1^k \sum (\lambda_i/\lambda_1)^k \beta_i v_i = 
		\lambda_1^k (\beta_1 v_i + \eps^{(k)}), \\
\end{eqnarray*}
donde $\eps$ tiende a $0$. ($\lambda_1$ es el autovalor mas grande en valor
absoluto). Sea $x^{(k)} = A^k x$. Sea $f$ continua tal que 
$f(\alpha x) = \alpha f(x)$ y $f(\beta_1 v_1) \neq 0$. Entonces
$$\limite{k}{\infty} f(x^{(k+1)}) / f(x^{(k)}) = \lambda_1 
  f(\beta_1 v_1 + \eps^{(k+1)}) / f(\beta_1 v_1 + \eps^{(k+1)}) = \lambda_1.$$
  
\textbf{M'etodo de la potencia inversa}. (Sirve si $\lambda_n$ es el autovalor
de menor valor absoluto y distinto de cero de $A$ no singular). 
Si $\lambda$ es autovalor de $A$, 
$\lambda^{-1}$ lo es de $A^{-1}$. $Ax = \lambda x \lthen  A^{-1} Ax = 
\lambda A^{-1} x \lthen x = \lambda A^{-1} x \lthen \lambda^{-1} x = A^{-1} x$.
Utilizando $x^{(k+1)} = A^{-1} x^k$ se puede demostrar que 
$f(x^{(k+1)}) / f(x^{(k)})$ tiende a $1/\lambda_n$.

\textbf{Encontrar todos los autovalores}. Una vez que encontre $\lambda_1$ me
busco $Q$ ortonormal tal que $QAQ^{-1}$, que tiene los mismos autovalores que
$A$, tenga todos ceros en la primer columna por debajo de la diagonal y 
$\lambda_1$ en la esquina superior izquierda (ver factorizaci'on QR). De este
modo, la submatriz $A'$ que resulta de tachar la primer fila y columna tiene
todos los autovalores excepto $\lambda_1$, as'i puedo obtener el segundo
m'as grande y repetir.

\label{theend}
\end{document}
